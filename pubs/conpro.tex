\documentclass[conference]{IEEEtran}
% correct bad hyphenation here
%\hyphenation{op-tical net-works semi-conduc-tor}

\usepackage{hyperref, url, cite}

\begin{document}
\title{horcrux-manager: Securing Password Management with Secret Sharing and
Autofill Defenses}
\dnote{prefer a more pronouncable title - is the "-manager" necessary?  Horcrux: A Password Manager for Paranoids}

\author{\IEEEauthorblockN{Samuel Havron
\IEEEauthorblockA{
\emph{University of Virginia}\\
\href{mailto:havron@virginia.edu}{havron@virginia.edu}\\}
}

\dnote{check call if submissions are supposed to be anonymous}
\dnote{additional authors}
}
\maketitle

\begin{abstract}
The abstract. Work-In-Progress Note: not all citations are proper yet or
  included in references! See the bib file for most references.
\end{abstract}

\section{Introduction and Motivation}

some references to be used:
\cite{silver2014password}
\cite{ion2015no}
\cite{li2014emperor}
\cite{goodin2015}
\cite{das2014tangled}
\cite{khandelwal2016}
\cite{titcomb2015}
\cite{shamir1979share}


The web is full of mixed advice on creating and managing passwords for consumers'
online accounts \dnote{for papers like this, just use \cite{gaw}, only need the cite number in text} (Gaw, 2006), and users often ignore frustrating, burdensome
security advice for rational reasons (Herley, 2009). Coupled with the risk that
recent major password database leaks pose to users' security across websites
with reused passwords (Roberts, 2016; LLV, 2016), there are many practical
problems of secure password creation and management by users that have yet to be
solved.

\dnote{this sentence belongs elsewhere: Users are often asked to remember many different passwords for their online
accounts, with an estimated 43-51\% of Internet users reusing the same password
across multiple sites due to password fatigue (Das, 2014).}

Password Managers (PMs), while introducing new security problems of their own,
offer a solution to password fatigue for users and promise
to eliminate the practice of insecure, reusable passwords across different
websites.  Password managers such as LastPass, Dashlane, and 1Password \dnote{add cites to PW managers}
require that the user remember one master password, which serves as
the key to an encrypted file containing all of the user's online account
passwords. Actual account passwords are typically generated randomly by the password manager
itself.

However, Ion et. al \cite{ion2015no} found
that many general users did not consider password managers as an important, or
even trusted security practice to stay safe online: ``password managers were
regarded with skepticism by non-experts, who instead preferred to remember
passwords, partly because, as one participant said, 'no one can hack my mind.'''.
Yet the same study found that computer security ``experts'' reported using a
password manager as one of the most important practices a user can do to
maintain online safety.

\dnote{next paragraph seems out of place here - need to figure out what is the main motivation - seems like a better story would focus on need to trust the PW manager for both the client-side software, and generation/storage of encrypted passwords, and use incidents of PW manager vulnerabilities compromises to illustrate this}

Many cloud-based PMs rely on client-side encryption/decryption and other strong
security practices to protect their users' data against autofill vulnerabilities
(such as XSS and CSRF attacks) and server side defenses (often with hosting on
Amazon Web Services). However, recent compromises to major password managers
(Khandelwal, 2016; Titcomb, 2015; and Goodlin, 2015) pose concerns for users who
may begin to lose trust in their PM provider, or perhaps already have. In this
paper, we describe a new password management system with stronger security
guarantees for cloud-based
PMs on both the client- (autofill policies) and server- (portability/storage
policies) side.  \dnote{make the key story about not trusting any one entity to provide client-side code and store/generate passwords - and goal is to minimize the code that must actually be trusted to a very small, auditable program. My hope would be the actual code someone needs to trust is small enough to include in an appendix to the paper (< 1 page), and there is a clear explanation why that is the only code that needs to be trusted.}

We introduce `horcrux-manager` (Horcrux) \dnote{add a footnote explaining where the name comes from}, a new password manager which
does not rely on trusting any single cloud provider with the users' (encrypted) passwords \dnote{the auto-fill focus seems secondary to the shinking code to audit, but can motivate the design, that also has benefits in preventing auto-fill vulnerabilities} 
nor does it trust any website's form submission to properly protect the user
from autofill vulnerabilities. "horcrux-manager" is available \dnote{hopefully it is true to call it "available" not just developed} as an open source
Firefox extension.
\dnote{cut the part about modern features unless there is something specific you have in mind for this worth mentioning}
%, and continues to provide numerous features of
%a modern password manager that users have come to expect, in addition to its
%security improvements.

\dnote{could use a short contributions paragraph here, that doubles as a roadmap to the paper}

\section{Threat Model and Assumptions}
The threat model for HM is an adversary who is assumed to steal encrypted
credentials from a variety of databases belonging to the user (less than a
threshold value $t$ as described in section 3), and can execute numerous
autofill attacks such as XSS/CSRF against a given domain a user is visiting. 
We assume that any credentials stolen could be unencrypted, and attempted to
recombine with the proposed secret-sharing scheme.

On the part of the user, we expect careful selection and variety in database
keystores chosen for their credentials, as well as use of a strong master password.
If the adversary can obtain or brute-force the users' local token and their
master password, the HM will undoubtedly be compromised. However, attacks on
database credentials and autofill vulnerabilities are much more difficult for an
adversary to compromise.

Most commercial password managers rely on Amazon Web Services or company servers 
to store encrypted user credentials. We assume that those who do not are not
using secret sharing as part of their proprietary software.

\section{Protecting Consumers from Cloud Storage Theft}
HM is unique in that it does not trust any one database store to keep a user's
encrypted passwords. Instead, HM splits passwords across multiple databases from
an arbitrary amount of cloud providers using ``horcruxes''; horcruxes are pieces
(``shares'') of each of the user's passwords for a given website, split into
shares using Shamir's Secret Sharing \cite{shamir1979share}. HM distributes $n$
shares (``horcruxes'') of a given credential across $n$ servers specified by the
end-user (e.g. AWS, Azure, personal server), requiring a security parameter of
exactly $\{k \mid 0 < k\leq n\}$ horcruxes to reconstruct the credential when
requested (a request requires knowing a master password and possessing a local
token). 

Using a smaller threshold could help improve the speed of retrieval, as well as
allow the user to still reconstruct passwords in the event of any keystore being
compromised or taken offline.

\section{Protecting Consumers from Autofill Vulnerabilities}
The proposed password manager also includes a means to prevent many common
autofill vulnerabilities from occurring while still offering autofill capability
for users, by using an ``auror''. The auror is a network traffic analysis tool
that runs as Javascript in the browser as part of the PM extension; upon
visiting a website with a login form, the auror checks whether or not the
website relies on client-side encryption before sending the user's credentials. 
The auror then populates the form fields with ``dummy''
credentials and waits for the user to submit the form, replacing the
dummy credentials with the real username and password once the connection
between the browser and the website server is secured over TLS, and JavaScript
cannot modify or learn the password as it is sent out over the network. Many
autofill vulnerabilities, such as those described in (Li, Z., 2014, and Silver,
2014) can be avoided by implementing the ``auror'' and not allowing the user's
real password to be sent across network traffic until the last possible moment,
as many autofill (generally XSS and CSRF) attacks will only gain the useless
dummy credentials.

\section{Implementation and Tests}
\subsection{Firefox Extension}
Code for `horcrux-manager' (HM) is available at \url{https://git.io/hcx-mgr}.
Compared to popular commercial offerings, what HM lacks in aesthetics and to
some extent, usability, it more than makes up for in protecting consumers from
the major security and privacy problems which LastPass, Dashlane, 1Password, and
others leave at risk.

\subsection{Tests}
See the `tests' directory of our code for output logs. We tested HM on several
popular websites for functionality using keystores in Amazon Web Services, as
well as for successfully evading an XSS attack outlined by (source). Password
reconstruction from $t$ shares was tested ...

\section{Future Work}
There are a number of improvements that can be made to HM in order to add to its
usability and overall security. Adding support for keystores on Microsoft Azure
and Google Cloud Platform would provide a stronger variety of sources for a
consumer to trust when distributing their credentials. Actual testing of
commercial password managers and HM for vulnerabilities as in \cite{silver2014password}
would further validate the project's viability as a usable tool for consumers.

\section{Conclusion}
The conclusion.

\bibliographystyle{plain}
\bibliography{references}

\end{document}
