\documentclass[conference]{IEEEtran}
% correct bad hyphenation here
%\hyphenation{op-tical net-works semi-conduc-tor}

\usepackage{hyperref, url, cite}

\begin{document}
\title{horcrux-manager: Securing Password Management with Secret Sharing and
Autofill Defenses}

\author{\IEEEauthorblockN{Samuel Havron
\IEEEauthorblockA{
\emph{University of Virginia}\\
\href{mailto:havron@virginia.edu}{havron@virginia.edu}\\}
}

}
\maketitle

\begin{abstract}
The abstract. Work-In-Progress Note: not all citations are proper yet or
  included in references! See the bib file for most references.
\end{abstract}

\section{Introduction and Motivation}

some references to be used:
\cite{silver2014password}
\cite{ion2015no}
\cite{li2014emperor}
\cite{goodin2015}
\cite{das2014tangled}
\cite{khandelwal2016}
\cite{titcomb2015}
\cite{shamir1979share}


The web is full of mixed advice on creating and managing passwords for consumers'
online accounts (Gaw, 2006), and users often ignore frustrating, burdensome
security advice for rational reasons (Herley, 2009). Coupled with the risk that
recent major password database leaks pose to users' security across websites
with reused passwords (Roberts, 2016; LLV, 2016), there are many practical
problems of secure password creation and management by users that have yet to be
solved.

Users are often asked to remember many different passwords for their online
accounts, with an estimated 43-51\% of Internet users reusing the same password
across multiple sites due to password fatigue (Das, 2014). Commercial password
managers (PMs) such as LastPass, Dashlane, and 1Password attempt to reduce this
burden by only asking the user to remember one master password, which serves as
the key to an encrypted file containing all of the user's online account
passwords, which are typically generated randomly by the password manager
itself.

Password Managers (PMs), while introducing new security problems of their own,
offer a solution to password fatigue for users and promise
to eliminate the practice of insecure, reusable passwords across different
websites. However, Ion et. al \cite{ion2015no} found
that many general users did not consider password managers as an important, or
even trusted security practice to stay safe online: ``password managers were
regarded with skepticism by non-experts, who instead preferred to remember
passwords, partly because, as one participant said, 'no one can hack my mind.'''.
Yet the same study found that computer security ``experts'' reported using a
password manager as one of the most important practices a user can do to
maintain online safety.

Many cloud-based PMs rely on client-side encryption/decryption and other strong
security practices to protect their users' data against autofill vulnerabilities
(such as XSS and CSRF attacks) and server side defenses (often with hosting on
Amazon Web Services). However, recent compromises to major password managers
(Khandelwal, 2016; Titcomb, 2015; and Goodlin, 2015) pose concerns for users who
may begin to lose trust in their PM provider, or perhaps already have. In this
paper, we describe a new password management system with stronger security
guarantees for cloud-based
PMs on both the client- (autofill policies) and server- (portability/storage
policies) side.

We introduce `horcrux-manager` (HM), a new password manager which
does not rely on trusting any single cloud provider with the users' passwords,
nor does it trust any website's form submission to properly protect the user
from autofill vulnerabilities. "horcrux-manager" is developed as a
Firefox web extension, and continues to provide numerous features of
a modern password manager that users have come to expect, in addition to its
security improvements.

\section{Threat Model and Assumptions}
The threat model for HM \dnote{don't like this acronym - prefer to just use Horcrux} is an adversary who is assumed to steal encrypted
credentials from a variety of databases belonging to the user (less than a
threshold value $t$ as described in section 3), and can execute numerous
autofill attacks such as XSS/CSRF against a given domain a user is visiting.  \dnote{this is a pretty confusing way to define the threat model.  Threat #1 is adversary who acquires password manager's database, and can guess user's master password; threat #2 is adversary who attacks user without access to PM database, using access point, web, etc. attacks (need to define this one more carefully); threat #0 is a malicious, incompetent, or compromised password manager operator, who can client-side code.  Try to make up a table that shows how different password methods are resistent to different treats (including no password manager, etc.)}

We assume that any credentials stolen could be unencrypted, and attempted to
recombine with the proposed secret-sharing scheme. \dnote{? don't understand this or its relevant}

On the part of the user, we expect careful selection and variety in database
keystores chosen for their credentials, as well as use of a strong master password.
If the adversary can obtain or brute-force the users' local token and their
master password, the HM will undoubtedly be compromised. \dnote{really? I thought your goal is to provide something better than standard PMs - if your assumption is user has strong password, and that adversary can win if they can brute-force user's local token/password (not sure how these are different), that doesn't seem like enough of a win. I would want there to be a design that means the attacker cannot do any kind of useful attack on local password/token without access to all the PW databases, but not sure if your design has this property.}
However, attacks on
database credentials and autofill vulnerabilities are much more difficult for an
adversary to compromise.

Most commercial password managers rely on Amazon Web Services or company servers 
to store encrypted user credentials. We assume that those who do not are not
using secret sharing as part of their proprietary software. \dnote{?  not getting the point of this paragraph}

\section{Protecting Users from Cloud Storage Theft}
HM is unique in that it does not trust any one database store to keep a user's
encrypted passwords. Instead, HM splits passwords across multiple databases from
an arbitrary amount of cloud providers using ``horcruxes'' \dnote{are you using horcruxes here to mean anything other than standard secret shares?  if so, let's avoid introducing a new confusing term here - it is just a secret share, which is well understood by most in this community}; horcruxes are pieces
(``shares'') of each of the user's passwords for a given website, split into
shares using Shamir secret sharing~\cite{shamir1979share}. HM distributes $n$
shares (``horcruxes'') of a given credential across $n$ servers specified by the
end-user (e.g. AWS, Azure, personal server), requiring a security parameter of
exactly $\{k \mid 0 < k\leq n\}$ horcruxes to reconstruct the credential when
requested (a request requires knowing a master password and possessing a local
token). 

Using a smaller threshold could help improve the speed of retrieval, as well as
allow the user to still reconstruct passwords in the event of any keystore being
compromised or taken offline.

\section{Protecting Consumers from Autofill Vulnerabilities}
\dnote{can real goal of this be to make a small, auditable client side?
explain that most of the complexity of a PM is doing the autofill, and it is also vulnerable to various threats, so main idea is to move all of that complexity out of the trusted computing base.  All that needs to be audited is that part that obtains and combines the shares, and this is made as small and simple as possible by doing it after "dummy" password is intercepted in outgoing network traffic.}
The proposed password manager also includes a means to prevent many common
autofill vulnerabilities from occurring while still offering autofill capability
for users, by using an ``auror'' \dnote{cite? is this something standard, or are you making up a new name for it? isn't it just using a Firefox extension API that allows hooking into the traffic?}. The auror is a network traffic analysis tool
that runs as Javascript in the browser as part of the PM extension; upon
visiting a website with a login form, the auror checks whether or not the
website relies on client-side encryption before sending the user's credentials. 
The auror then populates the form fields with ``dummy''
credentials and waits for the user to submit the form, replacing the
dummy credentials with the real username and password once the connection
between the browser and the website server is secured over TLS, and JavaScript
cannot modify or learn the password as it is sent out over the network. Many
autofill vulnerabilities, such as those described in (Li, Z., 2014, and Silver,
2014) can be avoided by implementing the ``auror'' and not allowing the user's
real password to be sent across network traffic until the last possible moment,
as many autofill (generally XSS and CSRF) attacks will only gain the useless
dummy credentials.

\section{Implementation and Tests}
\subsection{Firefox Extension}
Code for `horcrux-manager' (HM) is available at \url{https://git.io/hcx-mgr}.
Compared to popular commercial offerings, what HM lacks in aesthetics and to
some extent, usability, it more than makes up for in protecting consumers from
the major security and privacy problems which LastPass, Dashlane, 1Password, and
others leave at risk.

\dnote{important thing about the implementation is how small the part that needs to be audited is, include lines of code numbers, and clear explanation of design and where the trusted part is}

\subsection{Tests}
See the `tests' directory of our code for output logs. We tested HM on several
popular websites for functionality using keystores in Amazon Web Services, as
well as for successfully evading an XSS attack outlined by (source). Password
reconstruction from $t$ shares was tested ...

\dnote{are there any results to include? main question is if the latency of needing to combine shares is noticible to users, and how much it is}

\section{Future Work}
There are a number of improvements that can be made to HM in order to add to its
usability and overall security. Adding support for keystores on Microsoft Azure
and Google Cloud Platform would provide a stronger variety of sources for a
consumer to trust when distributing their credentials. Actual testing of
commercial password managers and HM for vulnerabilities as in \cite{silver2014password}
would further validate the project's viability as a usable tool for consumers.
\dnote{focus on more important issues than just supporting platforms - how to increase trust and make it more meaningful}

\section{Conclusion}
The conclusion.

\bibliographystyle{plain}
\bibliography{references}

\end{document}
